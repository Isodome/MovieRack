\section{Einleitung}

Die Architektur der Software orientiert sich hauptsächlich am Architekturstil Model-View-Controller. Dadurch wird erreicht, dass die Darstellung von Daten, die Datenhaltung und die Ablaufskontrolle/-logik in verschiedenen Modulen gekapselt werden kann. Die ersten drei Abschnitte geben einen genauen Einblick in diese drei Module, ihre Aufgaben und ihre Realisierung in diesem Projekt. Die folgenden zwei Abschnitte behandeln die Schnittstellen, über die das Programm Daten von Außen bezieht. Dies ist zum einen das AR-Modul, das mit Hilfe einer Videokamera Bilder der Außenwelt erhält und diese nach Sensorknoten durchsucht. Zum anderen die Gateway-Schnittstelle, die dafür zuständig ist, mit den Sensorknoten zu kommunizieren und aktuelle Messwerte zu empfangen und weiterzugeben. In den letzten Abschnitten wird anhand von einigen beispielhaften Sequenzdiagrammen der Ablauf der Kommunikation zwischen den Modulen dargestellt und erklärt und eine Gesamtübersicht über alle Klassen gegeben.
