\section{Funktionale Anforderungen}

\subsection{Ansichten}
\begin{description}
	\item[/FA10/] Auswahl verschiedener Ansichten
	\begin{description}
		\item[/FA11/] \textbf{Fiminfoansicht}: Darstellung einer Liste der Filme, am linken Programmrand, rechts davon Informationen des makierten Films.
		\item[/FA12/] \textbf{Listenansicht}: Eine Liste aller Filme mit einer kleinen Infoleiste am linken Programmrand
		\item[/FA13/] \textbf{Statistikansicht}: Ansicht in der man sich verschiedene Statisiken der gesamten Datenbank anzeigen lassen kann
		\item[/FA14/] \textbf{Personenansicht}: Ansicht zu einem Person in der Datenbank. In dieser werden detailierte Informationen zu einer Person angezeigt.
		\item[/FA15/] \textbf{Optionenansicht}: Hier sind alle Optionen in verschiedene Tabs gekapselt. Auch das erstellen neuer Listen(Listen bestehen in dem Fall aus einem Teil von Filmen)
		\item[/FA16/] \textbf{Film/Schauspieler-Bearbeitungsansicht}: Ein Pop-Up über dem man alle Einstellungen per Hand ändern kann.
		\item[/FA17/] \textbf{MultipleUpdate}: Ansicht in der Man eine bestimmte Anzahl oder auch alle Filme Updaten kann.
		\item[/FA17/]\textbf{MultipleAdd}: Ansicht, in der man mehrere Filme auf einmal hinzufügen kann.
		\item[/FA17/][optional] \textbf{VirtualFlipansicht}: Ansicht, in der die Cover in einer Reige angeornet sind und man durchscollenkann
		\item[/FA17/][optional] \textbf{Verleihmanagmentansicht}: Ansicht, in der man überblicken kann welche Filme man wem verliehen hat.
	\end{description}
	% \item Forumsvorschläge
	% \begin{itemize}
		% \item VirtualFlip
		% e\item einfache Filterfunktion (Als Anforderung und im Design)
		% \item Verleihmanagment
		% e\item Statistiken (Als Anforderung und im Design)
		% e\item Schauspieler Bereich (Als Anforderung und im Design)
		% \item Gruppierung von Filmen einer Reihe
		% e\item Empfehlungen \"Recommendation\" (Als Anforderung und im Design)
		% \item OFDb als Alternative auslesen
		% \item Scannen von Laufwerk-Verzeichnissen und den automatisierten Import der gefundenen Filmtitel, und übernahme des Verzeichnispfades
		% \item Unbekannte Filme sollen in einem Logfile gespeichert werden um direkt danak oder zu einem späteren Zeitpunkt den Import fortzusetzen. 
		% \item Doppelte Filme sollten übersprungen und ebenfalls in einem separaten Logfile, für spätere Bearbeitung, gespeichert werden.
		% \item Bearbeitete Log-\textbf{Einträge} nicht Logfiles sollten als erledigt markiert und erst auf bestätigen des Anwenders gelöscht werden.
		% \item Ferner wäre auch eine Import-Funktion aus einer Filmliste (z.B. Text-Datei) wünschenswert.
		% \item ! Ordner auswählen der bei jedem start überprüft wird und erkannt wird ob neue Film in ihm sind, und dann gefragt wird ob dese hinzugefügt werden sollen
		% e\item ! Metacritics (Als Anforderung und im Design)
		% \item Advanced Search http://www.imdb.com/search/title
		% \item Filmversion Directors Cut...
		% \item Rottentomatos
	% \end{itemize}
\end{description}


\newpage
\subsection{Fiminfoansicht}
\begin{description}
	\item[/FAXXX/] Anzeige einer einfache Liste
	\begin{description}
		\item[/FAXXX/] Anzeige des Covers in der Liste
		\item[/FAXXX/] Anzeige des Filmtitels in der Liste
		\item[/FAXXX/] Möglichkeit die Liste nach bestimten Kriterien Sortieren zu lassen
		\item[/FAXXX/] [optional] Anzeige einer weiteren einstellbaren Information des Films in der Liste
		\item[/FAXXX/] Möglichkeit einen Film hinzuzfügen
		\item[/FAXXX/] Möglichkeit einen Film zu löschen
		\item[/FAXXX/] Möglichkeit eine zu einem Film gehörende Datei abspielen zu lassen
		\item[/FAXXX/] Einheitliche Bar
		\begin{itemize}
			\item Titel, Orginaltitel, Erscheinungsjahr
			\item Rating auf IMDB mit Anzahl der Bewertungen
			\item IMDB Top 250 Platz
			\item Link zu IMDB
			\item Eigene Wertung (Eigene Wertung auf IMDB)
			\item Gernres
			\item Wertung auf Metacritics
			\item Link zu Metacritics
			\item Wertung auf Rottentomatos (Audience und Tomatoes)
			\item Link zu Rottentomatos
			\item Poster (möglichkeit auf Orginalgröße)
			\item Edit-Button um die Daten des Fims zu editieren
			\item [optional] Facebook Like-Button um den Film in Facebook zu veröffentlichen
		\end{itemize}
	\end{description}
	\item[/FAXXX/] Anzeige der Filminformationen auf der rechten Seite, mit verschiedenen Tabs
	\begin{description}
		\item[/FAXXX/] \textbf{Allgemeine Informationen Tab}
		\item Allgemeine Informationen zu einem Film. Diese Ansicht sollte den Großteil des am meisten Interessanten über einem Film enthalten.
		\begin{description}
			\item[/FAXXX/] Plot
			\item[/FAXXX/] Details (Editierbar, was angezeigt werden soll.)
			\begin{itemize}
				\item Laufzeit
				\item Premiere Datum
				\item Zuletzt gesehen
				\item Box Office
				\item ...
			\end{itemize}
			\item[/FAXXX/] Notes (Eigene Notizen)
			\item[/FAXXX/] Erweiterte Informationen (Eigends hinzugefügte Daten)
			\item[/FAXXX/] Produktion Personen (Director, Writer) mit link zur \textbf{Schauspieler Tab} (Doppelklick)
			\item[/FAXXX/] Top Actors mit link zur \textbf{Schauspieler Tab} (Doppelklick)	
		\end{description}			
		\item[/FAXXX/] \textbf{Erweitert Tab}
		\item Hier sollen weniger wichtigere Daten visualisiert werden.
			\begin{description}
				\item[/FAXXX/] Liste wann man alles den Film gesehen hat mit Textfeld für kurze Information
				\item[/FAXXX/] Liste wann man einen Film hinzugefügt hat und welches Format mit Textfeld für kurze Information
				\item[/FAXXX/] Übersicht der Awards wie auf IMDB
				\item[/FAXXX/] Also Kown As:
				\item[/FAXXX/] Memorable Quotes
				\item[/FAXXX/] Trivia
				\item[/FAXXX/] Empfehlungen anhand dieses Films, bei einem Doppelklick auf den Film soll sich die IMDB-Seite öffnen.
			\end{description}
		\item[/FAXXX/] \textbf{Schauspieler Tab}
		\item Die Ansicht in der alle Schauspieler des Films gezeigt werden.
			\begin{description}
				\item[/FAXXX/] Automatisch beim öffnen des Tabs wird der oberste Schauspieler(Star) angezeigt.
				\item[/FAXXX/] Liste aller Schauspieler des Films, mit Bild Name, Charakter und Alter zur Zeit des Films.
				\item[/FAXXX/] Regisseur
				\item[/FAXXX/] Autor
				\item[/FAXXX/] Falls ein Person makiert wird, passt sich der Bereich über die Person in dieser Ansicht dem Schauspieler an.
				\item[/FAXXX/] Bild des Schauspielers mit der Möglichkeit das Bild in Orginalgröße anzuzeigen. 
				\item[/FAXXX/] Schauspieler Details
					\begin{itemize}
						\item Geburtsdatum
						\item Alter
						\item Geburtsort
						\item Awards
						\item Durchschnittlicher IMDB Bewertung bzw. durchschnittliche eigene Bewertung aller Filme des Schauspielers in der Datenbank.
						\item Bekannt für Filme (Entnehmbar IMDB), bei einem Doppelklick auf einen dieser Filme, soll sich die IMDB-Seite dieses Films öffnen bzw die \textbf{Allgemeine Informationen Tab}-Seite des Films falls in der Datenbank vorhanden(optional)
						\item Lifetime Gross Total Box Office (entnehmbar Boxofficemjo)
						\item Durschnittlicher Boxoffice (entnehmbar Boxofficemojo)
						\item Liste aller Filme des Schauspielers in der Datenbank, bei einem Doppelklick auf einen dieser Filme soll zu diesem in die Filminfoansicht im \textbf{Allgemeine Informationen Tab} gewechselt werden.
					\end{itemize}
				\item[/FAXXX/] [optional] Andere Filme des Schauspielers welche nicht explizit in der Datenbank vorhanden sind.
			\end{description}
		\item[/FAXXX/] \textbf{Boxoffice Tab}
		\item Hier werden alle Informationen die mit den Einspielergebnissen zusammenhängen aufbereitet.
			\begin{description}
				\item[/FAXXX/] Einnahmen in Amerika
				\item[/FAXXX/] Einnahmen auserhalb Amerikas
				\item[/FAXXX/] Einnahmen Weltweit
				\item[/FAXXX/] Rang der Genres
				\item[/FAXXX/] Rang in Amerika der Filme aller Zeiten
				\item[/FAXXX/] Rang weltweit der Filme aller Zeiten
				\item[/FAXXX/] Einspielergebnis am Erscheinungswochenende
				\item[/FAXXX/] Wochenend Einspielergebnise mit Datum in Amerika
				\item[/FAXXX/] Einspielergebnisse geglieder nach Ländern.			
			\end{description}
		\item[/FAXXX/] \textbf{Awards Tab}
		\item Hier werden alle Preise bzw. Awards angezeigt.
			\begin{description}
				\item[/FAXXX/] Oscars Nominiert und Gewonnen, mit Kategorien und Nominierten
				\item[/FAXXX/] [optional] Eine beliebige Anzahl weiterer Preise mit der gleichen aufbereitung wie die Oscars.
			\end{description}
		\item[/FAXXX/] \textbf{Bilder Tab}
		\begin{description}
			\item Bilder des Films
			\item Möglichkeit Bilder hinzuzufügen und zu löschen
		\end{description}
	\end{description}
\end{description}


\newpage
\subsection{Listenansicht}
\begin{description}
	\item[/FAXXX/] Aufbereiten de Filme in einer Liste mit verschiedenen Spalten
	\item[/FAXXX/] Nach jeder Spalte soll sortiert werden können
	\item[/FAXXX/] Variable Anzahl an spalten und deren Belegung
	\item[/FAXXX/] Liste soll auch die Poster der Filme anzeigen
	\item[/FAXXX/] Am rechte Rand eine Infoleiste mit zwei Tabs.
	\item[/FAXXX/] Filminfo
	\begin{description}
		\item Falls ein Film makiert wurde
		\item[/FAXXX/] Poster
		\item[/FAXXX/] Klick auf das Poster soll es in Orginalgröße anzeigen
		\item[/FAXXX/] Filmtitel und Orignaltitel
		\item[/FAXXX/] Top Schauspieler
		\item[/FAXXX/] Allgemeine Informationen (Plot usw.)
		\item[/FAXXX/] Link zur \textbf{Filminfoansicht} dieses Films
		\item Falls ein Schauspieler makiert wurde
		\item[/FAXXX/] Kurzinformationen zum Film im Infotab
		\item[/FAXXX/] Filme des Schauspielers in der Datenbank
	\end{description}
	\item[/FAXXX/] Filter
	\begin{description}
		\item[/FAXXX/] Möglichkeit nach bestimmten Kriterien nur eine Auswahl an Filmen anzeigen zu lassen
		\begin{itemize}
			\item Genres
			\item Jahr
			\item Nicht gesehen seit
			\item Laufzeit
			\item Länder
			\item Formate
			\item Altersfreigaben
			\item ...
		\end{itemize}
	\end{description}
\end{description}


\newpage
\subsection{Statistikansicht}
\begin{description}
	\item Allgemeine Informationen zur Datenbank
	\begin{itemize}
		\item Anzahl Schauspieler, Regiseure Genres, Formate usw.
		\item Filme insgesamt
		\item Laufzeit insgesamt
		\item Durchschnittliche Wertung IMDB und eigene Wertung
		\item Insgesamt gesehen.
		\item Dauer der gesehenen Filme insgesamt
	\end{itemize}
	\item[/FAXXX/] Histogramm
	\begin{description}
		\item[/FAXXX/] Sammlunsverlauf
		\item[/FAXXX/] Hinzugefügt pro Moat
		\item[/FAXXX/] Gesehen pro Monat
		\item[/FAXXX/] Gesehen und Hinzugefügt pro Monat in einem Diagramm			
	\end{description}
	\item[/FAXXX/] Balkendiagramm
	\begin{description}
		\item[/FAXXX/] Top Schauspieler
		\item[/FAXXX/] Genres
		\item[/FAXXX/] Formate
		\item[/FAXXX/] Länder
		\item[/FAXXX/] Altersfreigaben
		\item[/FAXXX/] Erscheinungsjahre
	\end{description}
	\item[/FAXXX/] Kuchendiagramm
	\begin{description}
		\item[/FAXXX/] Top Schauspieler			
		\item[/FAXXX/] Top Genres
		\item[/FAXXX/] Formate
		\item[/FAXXX/] Länder
		\item[/FAXXX/] Altersfreigaben
		\item[/FAXXX/] Erscheinungsjahre
	\end{description}
\end{description}



\newpage
\subsection{Personenansicht}
\begin{description}
	\item[/FAXXX/] Suchleiste, mit Ergebnis einer Auswahl an Ergebnissen
	\begin{itemize}
		\item Suche über den Name einer Person
		\item Suche über die IMDB Nummer
		\item Suche über den Name eines Films
	\end{itemize}
	\item[/FAXXX/] Einheitliche Bar
	\begin{itemize}
		\item Alter
		\item Name
		\item Orginal Name
		\item Berufe
		\item Durchschnittliche Wertung der IMDB Filme in der Datenbank
		\item Durschnittliche Eigenwertung der Filme 
		\item Durschnittliche Wertung af Metacritics
		\item Best und Schlechtsbewertester Filme auf Rottentomatoes
	\end{itemize}
	\item[/FAXXX/] \textbf{Übersichts-Tab}
	\begin{description}
		\item[/FAXXX/] Schauspielerdetails
		\begin{itemize}
			\item Geburtstag
			\item Alter
			\item Geburtsort
			\item Awards
			\item IMDB Schnitt
			\item Einnahmen Gesamt
			\item Einnahmen Durschnitt
		\end{itemize}
		\item[/FAXXX/] Biographie
		\item[/FAXXX/] Filme für die die Person bekannt ist
		\item[/FAXXX/] Notizen
	\end{description}	
	\item[/FAXXX/] \textbf{Awards/Filme-Tab}
	\begin{description}
		\item[/FAXXX/] Übersicht über gewonnene Awards
		\item[/FAXXX/] Auflistung der Awards
		\item[/FAXXX/] Liste der Filme in der Datenbank mit diesem Person.
		\item[/FAXXX/]Wenn auf ein Fim geklickt wird, werden Indormationen dazu angezeigt
		\item[/FAXXX/] Bei Doppelklick kommt man zur \textbf{Filminfo-Übersicht des Films}
	\end{description}
	\item[/FAXXX/] \textbf{Bilder-Tab}
	\begin{description}
		\item Bilder einer Person
		\item Möglichkeit Bilder hinzuzufügen und zu löschen
	\end{description}
\end{description}

\newpage
\subsection{Optionenansicht}
\begin{description}
	\item[/FAXXX/] Automatisches Laden der Datenbank beim starten des Programms
	\item[/FAXXX/] Browserauswahl in welchem IMDB bzw. Links gestartet werden
	\item[/FAXXX/] Playerauswahl mit welchem eine Videodatei, gestartet werden soll Windowsstandart oder Programmpfad.
	\item[/FAXXX/] [optional] Einstellbaren Information des Films in der Liste
	\item[/FAXXX/] Filminfoansicht Details [/FAXXX/] (Editierbar, was angezeigt werden soll.)
	\item[/FAXXX/] Additional Information Werte eeintragen und erweitern können.
\end{description}

\subsection{Film/Schauspieler-Bearbeitungsansicht}
\begin{description}
	\item[/FAXXX/] Es müssen alle Daten eines Films bearbeitet werden können.
\end{description}

\subsection{MultipleUpdate}
\begin{description}
	\item[/FAXXX/] Auswahl an Informationen die akutalisiert werden sollen
	\item[/FAXXX/] Auswahl welche Filme aktualisiert werden sollen
	\begin{itemize}
		\item Nach erstellten listen
		\item Genres
		\item Letzter Aktualisierung
		\item Alle die keine IMDB Nummer haben
		\item Alle die eine IMDB Nummer haben
	\end{itemize}
\end{description}
		

