\section{GUI}

\subsection{GUI Textbeschreibung}
\begin{description}
	\item [Navigationsleiste]
			Die Navigationsleiste ist die am oberen Rand platzierte Leiste. Mit ihr kann man durch Tippen auf die jeweiligen Knöpfe durch die verschiedenen Ansichten wechseln. Sensorknoten, die in dieser Leiste platziert sind, können in der Detailansicht verglichen werden (vgl. Abb. \ref{detailansicht2.png}). Um einen Sensorknoten genauer zu betrachten, tippt man einfach darauf (vgl. Abb. \ref{erweitert.png}). Hinzu kommt noch das \glqq !\grqq , dieses blinkt rot, wenn ein Sensor seinen Grenzwert überschritten hat, und \glqq Standbild\grqq , wodurch das Bild in der Kameraansicht einfriert. D.h. alle Sensoren bleiben an der gleichen Stelle, wodurch man das Tablet z.B. hinlegen kann, ohne die Sensoren aus dem Blickfeld zu verlieren.
\end{description}


\newpage
\subsection{GUI Entwürfe}
%GUI Zustände
\begin{figure}[h]
	\begin{center}
	%\includegraphics[scale=1]{GUI_Entwuerfe/GUI_Zustaende.pdf}
	\end{center}
	\caption{Dieses Zustandsdiagramm verdeutlicht die 
		Übergänge zwischen den verschiedenen Hauptdarstellungsmodi.}
	\label{GUI_Zustaende.pdf}
\end{figure}

\clearpage