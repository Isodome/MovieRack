\section{Systemmodelle}
\subsection{Szenarien}
\subsubsection{Industrieanlagen-Wartung}
Bei einem Teil einer Industrieanlage besteht der Verdacht auf eine Fehlfunktion. Der für die Wartung vor Ort zuständige Inspekteur benötigt zur Beurteilung der Situation weitere Daten, die von den bereits vor Ort fest installierten Sensoren nicht oder nur unzureichend geliefert werden können. Der Inspekteur installiert nun ein Sensornetz bestehend aus über Funk kommunizierenden Sensorknoten und einem Gateway. Dieses Sensornetz sammelt Informationen über den Zustand des vermutlich defekten Teils der Anlage.\\Zur Beobachtung der Sensorinformationen verwendet der Inspekteur ein Tablet PC, das ihm ermöglicht ein Kamerabild der Anlage zu betrachten. Darauf sind die Positionen der automatisch lokalisierten Sensorknoten eingezeichnet. Außerdem bietet die Software Möglichkeit, die Messwerte der einzelnen Sensoren miteinander zu vergleichen und verschafft dem Inspekteur einen Überblick über die gemessenen Daten. Mithilfe von Diagrammen, die den zeitlichen Verlauf der Messreihen sowie eventuell auftretende Grenzwertüberschreitungen dokumentieren, kann die Fehlfunktion besser lokalisiert werden.
\clearpage